\documentclass{article}
\usepackage{blindtext}
\usepackage[a4paper, total={6in, 9.4in}]{geometry}

\usepackage{wrapfig}
\usepackage{graphicx}
\usepackage{mathtext}
\usepackage{amsmath}
\usepackage{siunitx} % Required for alignment
\usepackage{subfigure}
\usepackage{multirow}
\usepackage{rotating}
\usepackage{afterpage}
\usepackage[T1,T2A]{fontenc}
\usepackage[russian]{babel}
\usepackage{caption}
\usepackage[arrowdel]{physics}
\usepackage{booktabs}

\graphicspath{{pictures/}}

\title{Лабораторная работа 3.1.1: Магнитометр}
\author{Сидорчук Максим, Б01-304}
\date{\today}

\begin{document}

\pagenumbering{gobble}
\maketitle
\pagenumbering{arabic}

\textbf{Цель работы:} Определить горизонтальную составляющую магнитного поля Земли,
и установить количественное соотношение между единицами электрического тока в системах
СИ и СГС

\section{Измерение горизонтальной состовляющей магнитного поля Земли}
Параметры установки
\begin{align*}
    L &= 84.4\ см\\
    R &= 20\ см\\
\end{align*}
Параметры магнита
\begin{align*}
    m &= 2.870\ г\\
    l &= 24.3\ мм\\
    d &= 4.6\ мм\\
\end{align*}
Момент инерции магнита
\begin{equation*}
    J = \frac{ml^2}{12}\left[1+3\left({\frac{d}{2l}}\right)^2\right]=1.450\ г\cdotсм^2
\end{equation*}
Период колебания магнита в горизонтальной плоскости
\begin{equation*}
    T = \frac{29.13с}{16} = 1.82\ с
\end{equation*}
Смещение зайчика после вставки магнита в рамку
\begin{align*}
    x_1 &= 10.5\ см\\
    x_2 &= -9.8\ см\\
    \bar{x} = \frac{x_1 - x_2}{2} &= 10.15\ см\\
\end{align*}
Горизонтальное магнитное поле Земли
\begin{equation*}
    B_0 = \frac{2\pi}{TR}\sqrt{\frac{\mu_0 JL}{2\pi R\bar{x}}}=4.288\cdot10^{-5}\ [ед. СИ]
\end{equation*}
Горизонтальное магнитное поле земли по таблице:
\begin{equation*}
    B_0^{table} = 1.7*10^{-5} \ [ед. СИ]
\end{equation*}
\newpage
\section{Определение электродинамической постоянной}
Параметры установки
\begin{align*}
    N &= 34\\
    \nu &= 50\ Гц\\
    U &= 90\ В \\
    C &= 9 \cdot 10^5 см \\
    C_{СИ} &= 1.0918 \ мкФ \\
\end{align*}
Смещение зайчика после подачи тока
\begin{align*}
    x_1 &= 9.8\ см\\
    x_2 &= -10.4\ см\\
    \bar{x} = \frac{x_1 - x_2}{2} &= 10.1\ см\\
\end{align*}

Ток в системе СИ
\begin{equation*}
    I_{[СИ]} = \frac{2 B_0 R}{\mu_0 N} \cdot \frac{\bar{x}}{2L} = 4.93 \cdot 10^{-3}\ [ед. СИ]
\end{equation*}

Ток в системе СГС
\begin{equation*}
    I_{[СГС]} = CU\nu = 1.35 \cdot 10^7 [ед. СГС]
\end{equation*}

Электродинамическая постоянная
\begin{equation*}
    c\ \left[\frac{м}{с}\right] = \frac{1}{10} \frac{I_{[СГС]}}{I_{[СИ]}}=2.74\cdot10^8
\end{equation*}

\section{Выводы}
Получили значение электродинамической постоянной $c=2.74 \cdot 10^8\ м/с$, что отличется
от истинного значения $2.998 \cdot 10^8 м/с$ на 9\%.
\end{document}
