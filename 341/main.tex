\documentclass{lab}
\graphicspath{{pics/}}

\title {Лабораторная работа 3.4.1: Диа- и пара-магнетики}
\author {Сидорчук Максим Б01-304}
\date{\today}

\begin{document}
\maketitle

\section*{Краткие теоретические сведения}
Измерение магнитной восприимчивости материалов будем проводить с помощью расчета силы, действующей на магнетик в магнитном поле. При смещении образца на расстояние $ \Delta l $ внутрь магнитного поля магнитная сила, действующая на него, равна

\begin{equation}
    F = \left(\frac{\Delta W_m}{\Delta l}\right)_I,
\end{equation}
где $ \Delta W_m $ -- изменение магнитной энергии системы при постоянном токе
в обмотке электромагнита и, следовательно, при постоянной величине
магнитного поля в зазоре.\\
Магнитная энергия рассчитывается по формуле:
\[W_m=\frac{1}{2}\int (\mathbf{H}\mathbf{B})d\,V = \frac{1}{2\mu_0}\int\frac{B^2}{\mu}d\,V,\]
\begin{wrapfigure}[10]{l}{0.3\textwidth}
    \centering
    \includegraphics[height = 0.12\textheight]{energy.jpg}
\end{wrapfigure}

При смещении образца магнитная энергия меняется только в области зазора (в объёме площади $ S $ и высоты $ \Delta l $), а около верхнего конца стержня остаётся неизменной, поскольку магнитного поля там практически нет. Тогда изменение магнитной энергии будет:
\[\Delta W_m=\frac{1}{2\mu_0}\frac{(\mu B)^2}{\mu}S\Delta l - \frac{1}{2\mu_0}B^2 S\Delta l = (\mu - 1) \frac{B^2}{2\mu_0}S\Delta l \]

Следовательно, на образец действует сила
\[F = (\mu - 1)\frac{B^2}{2\mu_0}S = \chi\frac{B^2}{2\mu_0}S\]

Знак силы, действующей на образец, зависит от знака $ \chi $: образцы из парамагнитных материалов $( \chi  > 0)$ втягиваются в зазор электромагнита, а диамагнитные образцы $ (\chi < 0) $ выталкиваются из него.

\section*{Экспериментальная установка}

\begin{figure}
    \centering
    \includegraphics[width = 0.7\textwidth, height = 0.3\textheight]{ustanovka.png}
    \caption{Схема экспериментальной установки.}
\end{figure}

Магнитное поле создаётся в зазоре электромагнита, питаемого постоянным током. Диаметр полюсов существенно превосходит ширину зазора, поэтому поле в средней части зазора однородно. Величина тока, проходящего через обмотки электромагнита, задаётся регулируемым источником питания GPR и измеряется амперметром $ А $, встроенным в источник питания. Градуировка электромагнита (связь между индукцией магнитного поля $ B $ в зазоре электромагнита и силой тока $ I $ в его обмотках) производится при помощи милливеберметра либо тесламетра.\\
Сила, действующая на образец со стороны магнитного поля измеряется в помощью весов: смотрится разность веса образца вне поля и в поле.

\section*{Ход работы}

\subsection*{Градуировка электромагнита}

Сначала проведём градуировку магнита: с помощью тесламетра измеряем магнитное поле при разных токах:

\begin{table}[H]
    \begin{center}
        \begin{tabular}{|c|*{15}{c|}}
            \hline
            $I$, A & 0.1 & 0.2 & 0.3 & 0.4 & 0.5 & 0.6 & 0.7 & 0.8 & 0.9 & 1 & 1.1 & 1.17 \\\hline
            $B$, мТл & 110.6 & 210.3 & 303.1 & 399.2 & 490.4 & 592.4 & 682.5 & 785.3 & 918 & 1047.7 & 1130.4 & 1147.7 \\\hline
        \end{tabular}
        \caption{Результаты измерений}
    \end{center}
\end{table}

По полученным данным построим график зависимости $ B(I)$.

\begin{figure}[h!]
    \centering
    \includegraphics[width = \textwidth]{Graduation.png}
    \caption{Градуировка электромагнита $B(I)$}
\end{figure}

\subsection*{Измерение сил, действующих на образцы в магнитном поле}

При нулевом токе через электромагнит подвесим к весам один из образцов так, чтобы он не касался наконечников электромагнита. Обнулим показания весов, чтобы измерять непосредственно перегрузки $ \Delta P = F $ -- силы, действующей на образец при различных токах в обмотках электромагнита. Были получены следующие значения:\\
$m_{Al} = 25,228$ г, $m_{Cu} = 83,270$ г, $d = 1 \pm 0,01$ см (диаметр образцов).
\begin{table}[h!]
    \centering
    \begin{tabular}{|c|c|c|c|c|c|c|c|c|}
        \hline
        $I$, A & 0,22 & 0,30 & 0,45 & 0,60 & 0,75 & 0,90 & 1,05 & 1,15 \\ \hline
        Al, $\Delta m$, мг & 4 & 7 & 15 & 24 & 36 & 49 & 62 & 70 \\ \hline
        Cu, $\Delta m$, мг & -3 & -4 & -7 & -12 & -17 & -23 & -29 & -32 \\ \hline
    \end{tabular}
    \caption{Изменение веса образцов в магнитном поле}
\end{table}

\begin{figure}[h!]
    \centering
    \includegraphics[width = \textwidth]{graph.png}
    \caption{Зависимость $\Delta P(B^2)$ для меди и алюминия}
\end{figure}
\subsection*{Расчёт магнитной восприимчивости}
Построим графики зависимости $\Delta P(B^2)$ для меди и алюминия. Если $k$ - угловой коэффициент наклона графика, то
\[k = \frac{\chi S}{2\mu_0} \ \rightarrow \chi = \frac{2\mu_0 k}{S}\]
где $ S $ -- площадь поперечного сечения исследуемых образцов. В нашем случае $ S = (0,785 \pm 0,016) $ см$ ^2 $ для всех образцов.
Для меди значения для графика были взяты по модулю чисто для расчета, сам коэффициент будет отрицательным. Полученные значения представлены в таблице.

\begin{table}[h!]
    \centering
    \begin{tabular}{|c|c|c|c|}
        \hline
        Материал & $k$ & $\chi$ & $\varepsilon_{\chi}$ \\ \hline
        Алюминий Al & $5,34 \pm 0,022$ & $(1,71 \pm 0,79) \cdot 10^{-5}$ & 4,9\% \\ \hline
        Медь Cu & $2,41 \pm 0,009$ & $-(7,7 \pm 0,33) \cdot 10^{-6}$ & 4,7\% \\ \hline
    \end{tabular}
\end{table}
\section*{Выводы}

В ходе данной работы была измерена магнитная восприимчивость образцов меди и алюминия. Для алюминия и меди табличные значения удельной магнитной восприимчивости равны $ \chi_{Al.\text{уд}} = 0,61\cdot 10^{-9} \ \frac{\text{м}^3}{\text{кг}}$ и $ \chi_{Cu.\text{уд}} = -0,086 \cdot 10^{-9} \ \frac{\text{м}^3}{\text{кг}} $ соответственно, получается $\chi_{Al} = 1,64 \cdot 10^{-5}$ и $\chi_{Cu} = -7,7 \cdot 10^{-6}$, что достаточно близко к полученным значениям.

\end{document}
