\documentclass{article}
\usepackage{blindtext}
\usepackage[a4paper, total={6in, 9.4in}]{geometry}

\usepackage{wrapfig}
\usepackage{graphicx}
\usepackage{mathtext}
\usepackage{amsmath}
\usepackage{siunitx} % Required for alignment
\usepackage{subfigure}
\usepackage{multirow}
\usepackage{rotating}
\usepackage{afterpage}
\usepackage[T1,T2A]{fontenc}
\usepackage[russian]{babel}
\usepackage{caption}
\usepackage[arrowdel]{physics}
\usepackage{booktabs}

\graphicspath{{pictures/}}

\title{\begin{center}Лабораторная работа №3.1.1\end{center}
Магнитометр}
\author{Сидорчук Максим Б01-304}
\date{\today}

\begin{document}

\pagenumbering{gobble}
\maketitle
\pagenumbering{arabic}

\textbf{Цель работы:} Определить горизонтальную составляющую магнитного поля Земли,
и установить количественное соотношение между единицами электрического тока в системах
СИ и СГС

\section{Измерение горизонтальной состовляющей магнитного поля Земли}
Параметры установки
\begin{align*}
    L &= 102\ cм\\
    R &= 20\ см\\
\end{align*}
Параметры магнита
\begin{align*}
    m &= 5.861\ г\\
    l &= 4.00\ см\\
    d &= 0.49\ см\\
\end{align*}
Момент инерции магнита
\begin{equation*}
    J = \frac{ml^2}{12}\left[1+3\left({\frac{d}{2l}}\right)^2\right]=7.902\ г\cdotсм^2
\end{equation*}
Период колебания магнита в горизонтальной плоскости
\begin{equation*}
    T = \frac{217.1с}{20} = 10.855\ с
\end{equation*}
Смещение зайчика после вставки магнита в рамку
\begin{align*}
    x_1 &= 10.1\ см\\
    x_2 &= -9.1\ см\\
    \bar{x} = \frac{x_1 - x_2}{2} &= 9.6\ см\\
\end{align*}
Горизонтальное магнитное поле Земли
\begin{equation*}
    B_0 = \frac{2\pi}{TR}\sqrt{\frac{\mu_0 JL}{2\pi R\bar{x}}}=8.4\cdot10^{-6}\ [ед. СИ]
\end{equation*}

\newpage
\section{Определение электродинамической постоянной}
Параметры установки
\begin{align*}
    N &= 44\\
    \nu &= 50\ Гц\\
    U &= 90\ В = 0.3\ [ед. СГС]\\
    C &= 9 \cdot 10^5 см
\end{align*}
Смещение зайчика после подачи тока
\begin{align*}
    x_1 &= 9.8\ см\\
    x_2 &= -10.4\ см\\
    \bar{x} = \frac{x_1 - x_2}{2} &= 10.1\ см\\
\end{align*}

Ток в системе СИ
\begin{equation*}
    I_{[СИ]} = \frac{2 B_0 R}{\mu_0 N} \cdot \frac{\bar{x}}{2L} = 5.3 \cdot 10^{-3}\ [ед. СИ]
\end{equation*}

Ток в системе СГС
\begin{equation*}
    I_{[СГС]} = CU\nu = 1.4 \cdot 10^7 [ед. СГС]
\end{equation*}

Электродинамическая постоянная
\begin{equation*}
    c\ \left[\frac{м}{с}\right] = \frac{1}{10} \frac{I_{[СГС]}}{I_{[СИ]}}=2.65\cdot10^8
\end{equation*}

\section{Выводы}
Получили значение электродинамической постоянной $c=2.65 \cdot 10^8\ м/с$, что отличется
от истинного значения $2.998 \cdot 10^8 м/с$ на 12\%. Учитывая погрешности, связанные с
предварительной настройкой устройства, а так же условия, при которых проводились
измерения, считаю результаты эксперимента удовлетворительными в пределах погрешности.
\end{document}
